%%%%%%%%%%%%%%%%%%%%%%%%%%%%% Define Article %%%%%%%%%%%%%%%%%%%%%%%%%%%%%%%%%%
\documentclass[journal,transmag]{IEEEtran}
%%%%%%%%%%%%%%%%%%%%%%%%%%%%%%%%%%%%%%%%%%%%%%%%%%%%%%%%%%%%%%%%%%%%%%%%%%%%%%%

%%%%%%%%%%%%%%%%%%%%%%%%%%%%% Using Packages %%%%%%%%%%%%%%%%%%%%%%%%%%%%%%%%%%
\usepackage{geometry}
\usepackage{graphicx}
\usepackage{amssymb}
\usepackage{amsmath}
\usepackage{amsthm}
\usepackage{empheq}
\usepackage{mdframed}
\usepackage{booktabs}
\usepackage{lipsum}
\usepackage{graphicx}
\usepackage{color}
\usepackage{psfrag}
\usepackage{pgfplots}
\usepackage{bm}
\usepackage[spanish]{babel}
\usepackage[utf8]{inputenc}
\usepackage{float}
%%%%%%%%%%%%%%%%%%%%%%%%%%%%%%%%%%%%%%%%%%%%%%%%%%%%%%%%%%%%%%%%%%%%%%%%%%%%%%%

% Other Settings

%%%%%%%%%%%%%%%%%%%%%%%%%% Page Setting %%%%%%%%%%%%%%%%%%%%%%%%%%%%%%%%%%%%%%%
\geometry{a4paper}

%%%%%%%%%%%%%%%%%%%%%%%%%% Define some useful colors %%%%%%%%%%%%%%%%%%%%%%%%%%
\definecolor{ocre}{RGB}{243,102,25}
\definecolor{mygray}{RGB}{243,243,244}
\definecolor{deepGreen}{RGB}{26,111,0}
\definecolor{shallowGreen}{RGB}{235,255,255}
\definecolor{deepBlue}{RGB}{61,124,222}
\definecolor{shallowBlue}{RGB}{235,249,255}
%%%%%%%%%%%%%%%%%%%%%%%%%%%%%%%%%%%%%%%%%%%%%%%%%%%%%%%%%%%%%%%%%%%%%%%%%%%%%%%

%%%%%%%%%%%%%%%%%%%%%%%%%% Define an orangebox command %%%%%%%%%%%%%%%%%%%%%%%%
\newcommand\orangebox[1]{\fcolorbox{ocre}{mygray}{\hspace{1em}#1\hspace{1em}}}
%%%%%%%%%%%%%%%%%%%%%%%%%%%%%%%%%%%%%%%%%%%%%%%%%%%%%%%%%%%%%%%%%%%%%%%%%%%%%%%

%%%%%%%%%%%%%%%%%%%%%%%%%%%% English Environments %%%%%%%%%%%%%%%%%%%%%%%%%%%%%
\newtheoremstyle{mytheoremstyle}{3pt}{3pt}{\normalfont}{0cm}{\rmfamily\bfseries}{}{1em}{{\color{black}\thmname{#1}~\thmnumber{#2}}\thmnote{\,--\,#3}}
\newtheoremstyle{myproblemstyle}{3pt}{3pt}{\normalfont}{0cm}{\rmfamily\bfseries}{}{1em}{{\color{black}\thmname{#1}~\thmnumber{#2}}\thmnote{\,--\,#3}}
\theoremstyle{mytheoremstyle}
\newmdtheoremenv[linewidth=1pt,backgroundcolor=shallowGreen,linecolor=deepGreen,leftmargin=0pt,innerleftmargin=20pt,innerrightmargin=20pt,]{theorem}{Theorem}[section]
\theoremstyle{mytheoremstyle}
\newmdtheoremenv[linewidth=1pt,backgroundcolor=shallowBlue,linecolor=deepBlue,leftmargin=0pt,innerleftmargin=20pt,innerrightmargin=20pt,]{definition}{Definition}[section]
\theoremstyle{myproblemstyle}
\newmdtheoremenv[linecolor=black,leftmargin=0pt,innerleftmargin=10pt,innerrightmargin=10pt,]{problem}{Problem}[section]
%%%%%%%%%%%%%%%%%%%%%%%%%%%%%%%%%%%%%%%%%%%%%%%%%%%%%%%%%%%%%%%%%%%%%%%%%%%%%%%

%%%%%%%%%%%%%%%%%%%%%%%%%%%%%%% Plotting Settings %%%%%%%%%%%%%%%%%%%%%%%%%%%%%
\usepgfplotslibrary{colorbrewer}
\pgfplotsset{width=8cm,compat=1.9}
%%%%%%%%%%%%%%%%%%%%%%%%%%%%%%%%%%%%%%%%%%%%%%%%%%%%%%%%%%%%%%%%%%%%%%%%%%%%%%%
\hyphenation{op-tical net-works semi-conduc-tor}

\begin{document}

\title{Cálculo diferencial e integral, Análisis Numérico}
\author{José Danilo Bonilla Vides, 00019520}
\markboth{Cálculo diferencial e integral, Análisis Numérico}%
{Shell \MakeLowercase{\textit{et al.}}: Bare Demo of IEEEtran.cls for IEEE Transactions on Magnetics Journals}

\IEEEtitleabstractindextext{%
\begin{abstract}
The abstract goes here.
\end{abstract}


\begin{IEEEkeywords}
IEEE, IEEEtran, IEEE Transactions on Magnetics, journal, \LaTeX, magnetics, paper, template.
\end{IEEEkeywords}}

\maketitle

\IEEEdisplaynontitleabstractindextext

\IEEEpeerreviewmaketitle

\section{Introducción}
El cálculo de las matemáticas es una de las ciencias más importantes para el desarrollo humano. \\
Desde el principio de los tiempos han existido diferentes técnicas y métodos que han ayudado 
a la población humana a poder resolver problemas prácticos en dónde las matemáticas son una parte
fundamental del proceso de la solución. \\

Teóricamente pueden modelarse ciertas funciones para poder ser resueltas con el fin de un enfoque pedagógico y de aprendizaje
pero en el ámbito profesional y práctico de la vida real, muchas veces se presentan ciertas funciones 
que suponen un desafío y una complejidad que no se puede resolver con las soluciones algebraicas que son
las que devuelven un valor exacto y preciso de la función. \\

Es por eso que existen técnicas y métodos que son llamados "Métodos Numéricos" o también a veces llamado "Soluciones Numéricas" 
que permiten resolver problemas que involucran funciones que son demasiado complejas
o no tienen una solución exacta por cálculos elementales. \\

Los métodos numéricos, denominados así porque, usualmente, consisten
en realizar una sucesión más o menos larga de operaciones numéricas (normalmente mediante la ayuda de un
ordenador), al cabo de las cuales encontramos un valor numérico que, si bien no es la solución exacta del
problema, se le parece mucho, es decir, aproxima la solución buscada con una precisión razonablemente buena. \cite{apunteClase}
\\
En este reporte se abordarán 4 métodos numéricos pertenecientes a la rama del Cálculo diferencial e integral.
\\
Los cuatro métodos numéricos que se abordarán en este reporte son: 
\begin{itemize}
    \item \textbf{Método de Romberg}
    \item  \textbf{Método de Simpson}
    \item   \textbf{Cuadratura Gaussiana}
    \item   \textbf{Diferencia media}
\end{itemize}
Existen ciertas aplicaciones en las que se puedan utilizar estos métodos, pero en las que se enfocará este reporte serán las 4 siguientes:
\begin{itemize}
    \item \textbf{Enésima derivada en un punto particular}
    \item \textbf{Cálculo de áreas}
    \item \textbf{Cálculo de volúmenes}
    \item \textbf{Sólidos de revolución}
\end{itemize}

\section{Método de Romberg}
\subsection{Descripción}

La integración de Romberg es una técnica diseñada para obtener integrales numéricas
de funciones de manera eficiente. Se basa en aplicaciones sucesivas de la regla del trapecio. \cite{chapra_mtodos_2007}
\\
Werner Romberg (1909–2093) concibió este procedimiento para mejorar la precisión de la regla 
trapezoidal al eliminar términos sucesivos en la expansión asintótica en 1955. \cite{burden_numerical_2016}
\\

Este método requiere del uso de la regla del trapecio de aplicación múltiple, además de 2 estimaciones de la integral para obtener una tercera estimación más exacta. \\

Deducción: \cite{chapra_mtodos_2007} \\
Si partimos del hecho de que se requieren 2 estimaciones y se conoce que la estimación y el error correspondiente a la regla del trapecio de aplicación
múltiple se representa de manera general como:
\begin{equation}
    I = I(h) + E(h)
\end{equation}
Donde $I$ es la estimación de la integral, $h$ es el paso de integración, $I(h)$ es la estimación de la integral con el paso de integración $h$, y $E(h)$ es el error de truncamiento. \\

Entonces nuestras 2 estimaciones, usando tamaño de paso $h_1$ y $h_2$, se representan de manera general como:
\begin{equation}
    I(h_1)+ E(h_1) = I(h_2) + E(h_2)
\end{equation}

Además el error de la regla trapezoidal múltiple puede representarse en forma aproximada con la ecuación:

\begin{equation} 
    E \approxeq \frac{b-a}{12}h^2f'' 
\end{equation}

Vamos a suponer que $f''$ es constante para cada paso en $h$, así podemos determinar la razón entre los dos errores, de las dos estimaciones que necesitamos, lo cuál sería: 
\begin{equation}
    \frac{E(h_1)}{E(h_2)} \approxeq \frac{h_1}{h_2}
\end{equation}

Reagrupamos la ecuación 4 para poder sustituirla en la ecuación 2:
\begin{equation}
    E(h_1) \approxeq E(h_2)\frac{h_1}{h_2}^2
\end{equation}

Al sustituir nos queda:
\begin{equation}
    I(h_1)+  E(h_2) \frac{h_1}{h_2}^2  \approxeq I(h_2) + E(h_2)
\end{equation}

Despejando $E(h_2)$:
\begin{equation}
    E(h_2) \approxeq \frac{I(h_1)-I(h_2)}{1- (h_1/h_2)^2}
\end{equation}

A fin de obtener una mejor estimación de la integral, podemos sustituir en la ecuación 1:
\begin{equation}
    I \approxeq I(h_2)+\frac{1}{(h_1/h_2)^2-1}|I(h_2)-I(h_1)|
\end{equation}

Al demostrarse el error de esta estimación, da como resultado $O(h^4)$. Es un resultado mayor al del error de estimación al utilizar solo la regla del trapecio ($O(h^2)$).
Este resultado se ha logrado al utilizar 2 estimaciones $O(h^2)$. \\

Si utilizamos un caso donde el paso $h_2$ es dividido por la mitad de $h_1$, la ecuación se convierte en:
\begin{equation}
    I \approxeq I(h_2)+\frac{1}{2^2-1}|I(h_2)-I(h_1)|
\end{equation}

Al simplificar y agrupar términos de la ecuación, obtenemos:
\begin{equation}
    I \approxeq \frac{4}{3}I(h_2)-\frac{1}{3}I(h_1)
\end{equation}

Este procedimiento es un subconjunto de un método más general para combinar integrales y obtener mejores estimaciones. \\

Entonces podemos seguir reduciendo el paso $h$ sucesivamente para obtener mejores estimaciones, por ejemplo al dividir nuevamente por la 
mitad la ecuación anterior obtenemos una exactitud de $O(h^6)$, quedando la siguiente ecuación:
\begin{equation}
    I \approxeq \frac{16}{15}I_m-\frac{1}{15}I_l
\end{equation}

Así podemos ir dividiendo por la mitad el paso $h$ y la exactitud irá mejorando en cada iteración. \\

Existe una forma general para representar todo este proceso con una ecuación, y es atribuida a Romberg, la cuál se representa de la siguiente manera:
\begin{equation}
    I_{j,k} \approxeq \frac{4^{k-1}I_{j+1, k-1}-I_{j,k-1}}{4^{k-1}-1}
\end{equation}

También es llamado Algoritmo de Integración de Romberg. \\

Dónde:
\begin{itemize}
\item $I_{j+1, k-1}$ es la integral más exacta,
\item  $I_{j, k-1}$ es la menos exacta,
\item $I_{j,k}$ es la integral mejorada,
\item $k$ significa el nivel de la integración
\item  $j$ se usa para distinguir entre las dos estimación más y menos exacta.
\end{itemize}

Esta ecuación es de las más adecuadas para llevar su implemntación a la computadora.






\subsection{Demostraciones de convergencia}
    Al aplicar el método de Romberg, y utilizando el nivel 1 de integración es decir, en el momento que $k=1$,
    se obtiene una estimación de la integral de $f$ entre $a$ y $b$, de manera que el error de la estimación es $O(h^2)$.
    Es decir que al utilizar $k=1$, se está utilizando la regla del trapecio para estimar la integral. \\

    Como este método se basa en dividir por la mitad el paso h en cada iteración, y con la deducción hecha, al utilizar el nivel 2, es decir con $k=2$, y $j=1$,
    se obtiene la ecuación:
    \begin{equation}
        I_{1,2} \approxeq \frac{4}{3}I_{2, 1}-\frac{1}{3}I_{1,1}
    \end{equation}

    La cual tiene una convergencia de $O(h^4)$, y así al ir aumentando el nivel de la integración, se va aumentando la exactitud de la estimación en ($h^{2n}$).
    Es decir la convergencia será mucho mayor por cada aumento del nivel de integración. \\
\subsection{Análisis del error}
    El criterio de paro o terminación puede darse mediante una ecuación del error relativo porcentual, para detenerse cuándo se esté satisfecho con los resultados obtenidos.
    La ecuación con un método útil para calcular el error relativo porcentual es la siguiente:
    \begin{equation}
        |E_a| = |\frac{I_{1,k}-I_{1,k-1}}{I_{1,k}}|100\%
    \end{equation}


\subsection{Análisis de la eficiencia del método}
    Evidentemente al tener una convergencia que aumenta en factor de $h^{2n}$, el método de Romberg es muy eficiente. 
    Sobretodo al compararlo con otros métodos y obtener las cantidades de iteraciones necesarias para obtener un resultado que satisfaga
    el criterio de paro, podemos determinar la eficiencia del método.  \\

    A continuación se presenta el ejemplo 22.1 de la página 652, obtenido de \cite{chapra_mtodos_2007} \\
        
    Teniendo la función $f(x)=0.2 + 25x - 200x^2 + 675x^3 - 900x^4 +400x^5 $ y el intervalo $[0,0.8]$.
    Al evaluarse esa integral, por medio de aplicaciones simples y múltiples de la regla del trapecio, se obtuvieron los siguientes resultados:

    \begin{table}[h]
        \begin{center}
            \caption{Ejemplo Analisis de la eficiencia del método}
            \begin{tabular}{| c | c | c | c |}
                \hline
            Segmentos & h & 0.1728 & ${E_t\%}$ \\ \hline
            1 & 0.8 & 0.1728 & 89.5 \\
            2 & 0.4 & 1.0688 & 34.9 \\
            4 & 0.2 & 1.4848 & 9.5 \\ \hline
            \end{tabular}
            \label{tab:ejemplo}
        \end{center}
            \end{table}
 
        Se utiliza el metodo de Romberg, para obtener una estimación de la integral,
        recordamos que con el nivel 1, cuando $k=1$, se obtiene el mismo resultado que el de la regla de trapecio.
        Cuándo utilizamos el nivel 2 de integración, es decir cuando $k=2$, se obtiene la siguiente ecuación:
        \begin{equation}
            I_{1,2} \approxeq \frac{4}{3}I_{2, 1}-\frac{1}{3}I_{1,1}
        \end{equation}
        
        Obteniendo un resultado de $I_{1,2}$ = 1.367467. \\

        Al trabajar con la siguiente iteración se obtiene un resultado de $I_{1,3}$ = 1.640533 \\

        Al trabajar con la iteración de nivel 4, obtenemos exactamente el mismo resultado $I_{1,3}$, por lo que hemos llegado 
        a nuestra condición de paro. \\

        Hemos llegado a la condición de paro en tan solo 3 iteraciones
        con el método de Romberg, al compararlo con otros métodos cómo por ejemplo
        la regla de Simpson, se obtiene una eficiencia mucho mejor, ya que realizando el ejemplo
        por el método de Simpson, este se tardaría 256 segmentos en poder llegar al mismo resultado
        obtenido con Romberg. \\

        Incluso, al exigir un resultado exacto con 7 cifras significativas, Romberg solo tarda 15 evaluaciones en poder
        obtener el resultado exacto, demostrando que es muy eficiente y que possee una convergencia mucho más veloz
        que la regla del trapecio y la regla de Simpson.

\subsection{Ventajas y desventajas del método}
    \subsubsection{Ventajas}
    \begin{center}
    Ventajas del método de Romberg:
    \end{center}
    \begin{itemize}
        \item   Es un método con una convergencia mucho más rápida que los demás métodos, 
        por ejemplo al compararlo con el método de Simpson y la regla del trapecio
        \item   Su convergencia aumenta a factor de $h^{2n}$ por cada iteración, brindando resultados muy veloces
        \item   La adaptación a la computarización del método por medio de la 
        ecuación de la forma general presentada por Romberg es de una dificultad baja para el que lo implementa,
        sin importar el lenguaje y/o software utilizado.
        \item   El cálculo del error relativo como condición de paro es muy intuitivo, fácil y rápido
    \end{itemize}
    \subsection{Desventajas}
    \begin{center}
    Desventajas del método de Romberg:
    \end{center}
    \begin{itemize}
        \item   Depende de conocer 2 estimaciones de la integral que se está calculando
        \item   No se puede trabajar con él sin conocer la función de la integral que se está buscando evaluar
        \item   Si se desea trabajar a mano, requiere un arduo trabajo, ya que se necesita calcular 
        el nuevo resultado por la regla del trapecio al aumentar el nivel de integración, es decir se hace 
        un doble trabajo por cada nueva iteración realizada.
    \end{itemize}

\subsection{Seudo-código}
        Seudo-código del método de Romberg: \cite{in_mr} \\

        ENTRADA: Extremos a, b; entero n $>$ 0 \\
        SALIDA: un arreglo R (calcule R por renglones; solo los 2 últimos renglones
        se guardan en almacenamiento) \\
        PROCESO: \\
        \begin{itemize}
            \item   PASO 1; tomar el valor de $h=(b-a)$; \\
                        $R_{1,1} = \frac{h}{2} (f(a)+f(b))$; \\
            \item   PASO 2; SALIDA ($R_{1,1}$); \\
            \item   PASO 3; Para i = 2, ..., n hacer pasos 4-8; \\
            \item   PASO 4; Tomar $R_{2,1}$ = \\
            $[R_{1,1} + h\displaystyle\sum_{k=1}^{2^{i-1}} f(a)+(k-0.5)h]/2$; \\
            (Aproximación por la regla del trapecio) \\
            \item   PASO 5; Para j = 2,...,i; \\
                    Tomar $R_{2,j} = R_{2,j-1}+\frac{R_{2,j-1}-R_{1,j-1}}{4^{j-1}-1}$; \\
                    (Extrapolación) \\
            \item   PASO 6; SALIDA ($R_{2,j}$ por j= 1,2,...,j); \\
            \item   PASO 7; Tomar $h = \frac{h}{2}$; \\
            \item   PASO 8; Para j = 1,2,...,i; Tomar $R_{1,j}$ = $R_{2,j}$; \\
                    (Actualizar el renglón 1 de R) \\
            \item   PASO 9; PARAR;\\
        \end{itemize}


        Seudo-código más detallado: \cite{chapra_mtodos_2007} \\
        $FUNCTION$ $Romberg (a, b, maxit, es)$ \\
        $LOCAL$ $I(10, 10)$ \\
        $n = 1$ \\
        $I_{1,1} = TrapEq(n, a, b)$ \\
        $iter = 0$ \\
        $DOFOR$ \\
        $iter = iter + 1$ \\
        $n = 2^iter$ \\
        $l_{iter+1,1} = TrapEq(n, a, b)$ \\   
        $DOFOR$ $k = 2, iter + 1$ \\
        $j = 2 + iter – k$ \\
        $I_{j,k} = (4^{k–1} * I_{j+1,k–1} – I_{j,k-1})/(4^{k–1} – 1)$ \\
        $END$ $DO$ \\
        $ea = ABS((I_{1,iter+1} – I_{2,iter})/I_{1,iter+1}) * 100$ \\
        $IF$ $(iter \geq maxit $ OR $ ea \leq es)$ $EXIT$ \\
        $END$ $DO$ \\
        $Romberg = I_{1,iter+1}$ \\
        $END$ $Romberg$ 

\subsection{Ejemplos}
    Ejemplos del método de Romberg: \cite{in_mr}
    \subsection{Ejemplo 1}
    Evaluar con el método de Romberg, con un error máximo de 5 cifras significativas
    \begin{equation} 
        \int_1^2 f(x)=\frac{1}{x} dx
    \end{equation}
    \begin{enumerate}
    \item Supongamos que $T_{N,1}$ es la aproximación de la integral de $f(x)$ en el intervalo $[1,2]$ \\begin{align*}
        por medio de la regla trapezoidal con $n$ = $2^N$ subintervalos. Se tendrá la siguiente relación de recurrencia
        que permite calcular dicha aproximación: \\
        \begin{center}
            \begin{equation} \footnotesize
                T_{n,1} = \frac{1}{2}[T_{N-1,1} + \frac{{b-a}}{2^{N-1}}\displaystyle\sum_{{i=1 {\triangle i=2}}}^{{2^N-1}} f(a+(\frac{b-a}{2^N}i))]
            \end{equation}
        \end{center}
    \item  Los restantes términos de las distintas sucesiones se calculan mediante la fórmula
    general de extrapolación de Romberg: \\ 
    \begin{center}
        \begin{equation}
            T_{n,j} = \frac{4^{j-1}T_{N+1,j-1}-T_{N,j-1}}{4^{j-1}-1}
        \end{equation}
    \end{center}
    \item   A continuación se presenta la tabla con los valores calculados para los TN, j
    mediante (17) y (18): \\
    \begin{center}
        \begin{table}[h]
            \begin{center}
                \caption{Ejemplo 1}
                \begin{tabular}{| c | c | c | c | c |}
                    \hline
                N & J=1 & 2 & 3 & 4 \\ \hline
                0 & 0.750000 & 0.694445 & 0.693175  & 0.693148\\ \hline
                1 & 0.708334  & 0.693254  & 0.693148 & \\  \hline
                2 & 0.697024  & 0.693155  & 0.693146 &\\ \hline
                3 & 0.694122 & 0.693147  & &\\ \hline
                3 & 0.693391 &  & &\\ \hline
                \end{tabular}
                \label{tab:EjemploEjercicio1}
            \end{center}
                \end{table}
    \end{center}
    \item   Teniendo en cuenta que el valor exacto de la integral es Ln(2)=0.693147, las
    aproximaciones conseguidas se pueden considerar como buenas. Las pequeñas
    diferencias entres sucesiones son debidas al error de redondeo. 
    \end{enumerate}

    \subsection{Ejemplo 2}
    Usar el algoritmo de Romberg para calcular el valor de la integral de $f(x)$ usando segmentos de 1, 0.5 y 0.25 \\
    \begin{equation}
        \int_0^1 f(x)={e^{x^2}} dx
    \end{equation}
    \begin{enumerate}
    \item Se hace la integración de nivel 1, usando la regla del trapecio\\
        \begin{itemize}
            \item   $(h_1 = 1)$: \\
                \begin{equation}
                    I(h_1) = \frac{1-0}{2}[e^{0^2} + e^{1^2}] = 1.859140914
                \end{equation}
                \item   $(h_2 = 0.5)$: \\
                \begin{equation}
                    I(h_2) = \frac{1-0}{4}[e^{0^2} + 2e^{(\frac{1}{2})^2}+ e^{1^2}] = 1.571583165
                \end{equation}
                \item   $(h_3 = 0.25)$: \\
                \begin{equation}
                    I(h_3) = \frac{1-0}{8}[e^{0^2} + 2[e^{(\frac{1}{4})^2}+e^{(\frac{2}{4})^2}+e^{(\frac{3}{4})^2}] + e^{1^2}] = 1.490678862
                \end{equation}
        \end{itemize}
    \item   Ahora se pasa al segundo nivel de aproximación donde se usa la fórmula que se dedujo
    anteriormente:
    \begin{center}
        \begin{equation}
            \frac{4}{3}I(h_2)-\frac{1}{3}I(h_1)
        \end{equation}
    \end{center}

    Donde $I(h_1)$ es la integral menos exacta (la que usa menos subintervalos) e $I(h_2)$ es
    la más exacta (la que usa el doble de subintervalos). 

    \item   Utilizando $h1 = h1$ (calculado anteriormente) y $h2 = h2$ (calculado anteriormente) se obtiene:
    \begin{center}
    \begin{equation}
        \frac{4}{3}(1.571583165)-\frac{1}{3}(1.859140914) = 1.475730582
    \end{equation}
    \end{center}
    Utilizando $h1 = h2$ (calculado anteriormente) y $h2 = h3$ (calculado anteriormente) se obtiene:
    \begin{center}
    \begin{equation}
        \frac{4}{3}(1.490678862)-\frac{1}{3}(1.571583165) = 1.463710761
    \end{equation}
    \end{center}
    \item   Se continua con la integración de nivel 3, usando la ecuación:
    \begin{center}
    \begin{equation}
        \frac{16}{15}I_m-\frac{1}{15}I_l
    \end{equation}
    \end{center}
    Donde: $I_l$ es la integral menos exacta (la que usa menos subintervalos) e $I_m$ es la más exacta (la que usa el doble de subintervalos).
    \item   Se obtiene como resultado de la integración de nivel 3:
    \begin{center}
    \begin{equation}
        \frac{16}{15}(1.46...)-\frac{1}{15}(1.47...) = 1.46290944
    \end{equation}
    \end{center}
    Siendo un resultado bastante aproximado y preciso, si se desea obtener un resultado más exacto, se puede seguir subiendo el nivel de la integración.

    Por ejemplo para el nivel 4 de integración se utilizará la ecuación:
    \begin{center}
    \begin{equation}
        \frac{64}{63}I_m-\frac{1}{63}I_l
    \end{equation}
    \end{center}
    Donde: $I_l$ es la integral menos exacta (la que usa menos subintervalos) e $I_m$ es la más exacta (la que usa el doble de subintervalos).
    \end{enumerate}
    \section{Método de Simpson}
\subsection{Descripción}
    Teorema de Simpson
\subsection{Demostraciones de convergencia}
    Demostraciones de convergencia
\subsection{Análisis del error}
\begin{lipsum}
    Análisis del error
\end{lipsum}
\subsection{Análisis de la eficiencia del método}
\begin{lipsum}
    Análisis de la eficiencia del método
\end{lipsum}
\subsection{Ventajas y desventajas del método}
\begin{lipsum}
    Ventajas y desventajas del método
\end{lipsum}
\subsection{Pseudo-código}
\begin{lipsum}
    Pseudo-código
\end{lipsum}
\subsection{Ejemplos}
\begin{lipsum}
    Ejemplos
\end{lipsum}
\section{Cuadratura Gaussiana}
\subsection{Descripción}
\begin{lipsum}
    Teorema de Gauss
\end{lipsum}
\subsection{Demostraciones de convergencia}
\begin{lipsum}
    Demostraciones de convergencia
\end{lipsum}
\subsection{Análisis del error}
\begin{lipsum}
    Análisis del error
\end{lipsum}
\subsection{Análisis de la eficiencia del método}
\begin{lipsum}
    Análisis de la eficiencia del método
\end{lipsum}
\subsection{Ventajas y desventajas del método}
\begin{lipsum}
    Ventajas y desventajas del método
\end{lipsum}
\subsection{Pseudo-código}
\begin{lipsum}
    Pseudo-código
\end{lipsum}
\subsection{Ejemplos}
\begin{lipsum}
    Ejemplos
\end{lipsum}
\section{Diferencia Media}
\subsection{Descripción}
\begin{lipsum}
    Teorema de Diferencia Media
\end{lipsum}
\subsection{Demostraciones de convergencia}
\begin{lipsum}
    Demostraciones de convergencia
\end{lipsum}
\subsection{Análisis del error}
\begin{lipsum}
    Análisis del error
\end{lipsum}
\subsection{Análisis de la eficiencia del método}
\begin{lipsum}
    Análisis de la eficiencia del método
\end{lipsum}
\subsection{Ventajas y desventajas del método}
\begin{lipsum}
    Ventajas y desventajas del método
\end{lipsum}
\subsection{Pseudo-código}
\begin{lipsum}
    Pseudo-código
\end{lipsum}
\subsection{Ejemplos}
\begin{lipsum}
    An article \cite{anarticle}
\end{lipsum}
\newpage
\section{Bibliografía}
\bibliographystyle{IEEEtran}
\bibliography{Proyecto}
\end{document}

